% !TeX root=FastStrProof.tex

\section{اثبات قضیه بیز تغییراتی با استفاده از تقریب میدان میانگین}
\label{app:VBProof}

ایده‌ی اصلی روش‌های بیز تغییراتی (به اختصار
\lr{VB})
تقریب‌زدن
$\TrueCProb{\theta}{X}$
با تقریب‌زدن عبارات حاشیه‌ای به صورت زیر است
\cite{VBMethodsInSignal}:


\begin{equation*}
\TrueCProb{\theta}{X} \approx
	\BreveCProb{\theta}{X} =
		\BreveCProb{\theta_1}{X}
		\BreveCProb{\theta_2}{X}
\end{equation*}


در واقع تقریب‌زدن باعث می‌شود زیر مجموعه‌هایی از پارامترها به صورت مستقل
از هم در نظر گرفته شوند که توسط طراح الگوریتم به صورت افرازهایی از
$\theta$
انتخاب شده‌اند. تقریب بهینه با کمینه کردن واگرایی
$\BreveCProb{\theta}{X}$
نسبت به
$\TrueCProb{\theta}{X}$
با استفاده از ملاک واگرایی
KL
بدست می‌آید:


\begin{equation}
\label{eq:KLDfunc}
\StarCProb{\theta}{X} = \underset{
	\BreveCProb{\theta_1}{\centerdot}
	\BreveCProb{\theta_2}{\centerdot}
}{\arg \min} \KL{
		\BreveCProb{\theta_1}{\centerdot}
		\BreveCProb{\theta_2}{\centerdot}
	}{
		\TrueCProb{\theta}{X}
	}
\end{equation}


با بهینه‌سازی فرمول
\eqref{eq:KLDfunc}
، به توابع
$\BreveCProb{\theta_1}{X}$
و
$\BreveCProb{\theta_2}{X}$
با فرم شناخته‌شده‌ای خواهیم رسید که به آن‌ها حاشیه‌ای‌های
VB
گفته می‌شود. در روش‌های
VB
خانواده‌ای از توزیع‌ها در نظر گرفته می‌شود که به طور تقریبی به حاشیه‌ای‌های
ضرب شده در هم نزدیک باشد. روش‌های
VB
توازنی بین پیچیدگی محاسبات و دقت تقریب برقرار می‌کنند.



\begin{theorem}[بیز تغییراتی]
\label{thm:VarBayeseTheorem}
\cite{VBMethodsInSignal}
گیریم
$\TrueCProb{\theta}{X}$
پسینی پارامتر چند متغیره‌ی
$\theta$
باشد که به
$K$
زیر بردار به صورت
$\theta = \left[ \theta_1, \theta_2, \dots \theta_K \right]$
افراز شده باشد و
$\BreveCProb{\theta}{X}$
یک توزیع تقریبی محدود شده به توزیع‌های مستقل شرطی برای
$\theta_1, \theta_2, \dots, \theta_K$
به صورت زیر باشد:


\begin{gather}
\BreveCProb{\theta}{X} = \BreveCProb{\theta_1, \theta_2, \dots, \theta_K}{X} =
	\prod_{i=1}^{K} \BreveCProb{\theta_i}{X}					\nonumber \\
%%%%%%%%%%%%%%%%%%%%
\intertext{آنگاه کمینه‌
$KLD_{VB}$:}													\nonumber \\
%%%%%%%%%%%%%%%%%%%%
\StarCProb{\theta}{X} = \underset{
	\AppProb{ \cdot }
}{
	\arg \min
} \KLDiv{\BreveCProb{\theta}{X}}{\TrueCProb{\theta}{X}}			\nonumber \\
%%%%%%%%%%%%%%%%%%%%
\intertext{از عبارت زیر بدست می‌آید:}							\nonumber \\
%%%%%%%%%%%%%%%%%%%%
\StarCProb{\theta_i}{X} \propto
	\exp \left(
		\Exp{\StarCProb{\theta_{/i}}{X}}{\ln \Dns{f}{\theta, X}}
	\right), \qquad i = 1, \dots K
\label{eq:VBMargins}
\end{gather}


که در آن
$\theta_{/i}$
مجموعه‌ی پارامترهای مکمل
$\theta_i$
در
$\theta$
و
$\StarCProb{\theta_{/i}}{X} =
	\prod_{j = 1, j \neq i}^{K} \StarCProb{\theta_j}{X}
$
تعریف می‌شوند. به
$\StarCProb{\theta}{X}$،
\VBApprox
و به
$\StarCProb{\theta_i}{X}$
نیز
\VBMargin
ها می‌گوییم.
\end{theorem}
